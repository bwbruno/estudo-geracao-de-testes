\documentclass[10pt]{article}

\usepackage[brazil]{babel}
\usepackage[utf8]{inputenc}
\usepackage{tikz}
\usepackage{caption} 
\usepackage{subcaption} % for subfigures
\usepackage{geometry}
\usepackage{tabu}
\usepackage{float}
\usepackage[T1]{fontenc}
\usepackage{inconsolata}
\usepackage{color}
\usepackage{listings}
\usepackage{amsmath}
\usepackage{amssymb}
\usepackage{hyperref}
\usepackage{enumitem}
\usepackage{mathtools}
\usepackage{mdframed}
\usepackage{setspace}

\usepackage[scaled]{helvet}

\renewcommand\familydefault{\sfdefault}
% \renewcommand{\thesubsection}{\alph{subsection})}

\geometry{
    a4paper,
    total={170mm,257mm},
    left=20mm,
    top=20mm,
}

\onehalfspacing

\usetikzlibrary{arrows}

\begin{document}


\begin{figure}[!ht]
    \centering
    \subfloat{
    \includegraphics[height=2.4cm]{brasao-ufrn.png}
    \label{UFRN Logo}
    }
    \hspace{12.09cm}
    \subfloat{
    \includegraphics[height=2.2cm]{brasao-metropole.png}
    \label{DCA Logo}
    }
    %\caption{}
    \label{Logos}
\end{figure}

\vspace{-2.9cm}

\begin{center}

UNIVERSIDADE FEDERAL DO RIO GRANDE DO NORTE\\
INSTITUTO METRÓPOLE DIGITAL\\
\vspace{0.5cm}
DIM0512 - TESTE DE SOFTWARE II
\vspace{1.25cm}

\Large Comparando a Geração de Testes Randoop, Evosuite e ChatGPT
\vspace{0.25cm}

\vspace{1cm}
{\large{\bf Aluno: } Bruno Wagner Barbosa Rodrigues
}
\end{center}

\vspace{0.5cm}

\lstset{language=Java,
        basicstyle=\ttfamily\small,
        keywordstyle=\color{blue}\ttfamily,
        stringstyle=\color{orange}\ttfamily,
        commentstyle=\color{red}\ttfamily,
        morecomment=[l][\color{orange}]{\#},
        breaklines=true,
        extendedchars=true, % Enables extended characters
        literate=%
            {á}{{\'a}}1 {é}{{\'e}}1 {í}{{\'i}}1 {ó}{{\'o}}1 {ú}{{\'u}}1
            {Á}{{\'A}}1 {É}{{\'E}}1 {Í}{{\'I}}1 {Ó}{{\'O}}1 {Ú}{{\'U}}1
            {à}{{\`a}}1 {è}{{\`e}}1 {ì}{{\`i}}1 {ò}{{\`o}}1 {ù}{{\`u}}1
            {À}{{\`A}}1 {È}{{\`E}}1 {Ì}{{\`I}}1 {Ò}{{\`O}}1 {Ù}{{\`U}}1
            {ã}{{\~a}}1 {õ}{{\~o}}1 {Ã}{{\~A}}1 {Õ}{{\~O}}1
            {â}{{\^a}}1 {ê}{{\^e}}1 {î}{{\^i}}1 {ô}{{\^o}}1 {û}{{\^u}}1
            {Â}{{\^A}}1 {Ê}{{\^E}}1 {Î}{{\^I}}1 {Ô}{{\^O}}1 {Û}{{\^U}}1
            {ç}{{\c{c}}}1 {Ç}{{\c{C}}}1
}

\lstdefinestyle{Raw}{
        basicstyle=\ttfamily\small,
        keywordstyle=\color{blue}\ttfamily,
        morekeywords={Teste, retorna},
        stringstyle=\color{black}\ttfamily,
        commentstyle=\color{red}\ttfamily,
        morecomment=[l][\color{red}]{\#},
        breaklines=true,
        extendedchars=true, % Enables extended characters
        literate=%
            {á}{{\'a}}1 {é}{{\'e}}1 {í}{{\'i}}1 {ó}{{\'o}}1 {ú}{{\'u}}1
            {Á}{{\'A}}1 {É}{{\'E}}1 {Í}{{\'I}}1 {Ó}{{\'O}}1 {Ú}{{\'U}}1
            {à}{{\`a}}1 {è}{{\`e}}1 {ì}{{\`i}}1 {ò}{{\`o}}1 {ù}{{\`u}}1
            {À}{{\`A}}1 {È}{{\`E}}1 {Ì}{{\`I}}1 {Ò}{{\`O}}1 {Ù}{{\`U}}1
            {ã}{{\~a}}1 {õ}{{\~o}}1 {Ã}{{\~A}}1 {Õ}{{\~O}}1
            {â}{{\^a}}1 {ê}{{\^e}}1 {î}{{\^i}}1 {ô}{{\^o}}1 {û}{{\^u}}1
            {Â}{{\^A}}1 {Ê}{{\^E}}1 {Î}{{\^I}}1 {Ô}{{\^O}}1 {Û}{{\^U}}1
            {ç}{{\c{c}}}1 {Ç}{{\c{C}}}1
}


\fontsize{12}{12}\selectfont
\section{Repositório Git}

O código-fonte do projeto desenvolvido e analisado está disponível em:

\begin{quote}
\url{https://github.com/bwbruno/estudo-geracao-de-testes}
\end{quote}

\subsection{Estrutura de diretórios repositório}

\begin{mdframed}[backgroundcolor=gray!10]
\begin{lstlisting}
    .
    |-- geradores
    |   |-- evosuite
    |   |-- randoop
    |-- projeto
    |   |-- simplesrh
    |   |   |-- src
    |   |   |   |-- main
    |   |   |   |-- test-evosuite1
    |   |   |   |-- test-evosuite2
    |   |   |   |-- test-evosuite3
    |   |   |   |-- test-randoop1
    |   |   |   |-- test-randoop2
    |   |   |   |-- test-randoop3
    |   |   |-- target
    |-- relatorio
\end{lstlisting}
\end{mdframed}

\subsection{Projeto SimplesRH}

O projeto SimplesRH é um programa de linha de comando simples onde é possível executar algumas operações a partir de uma lista de funcionários em formato JSON.

\vspace{0.25cm}

\break
\subsubsection{Comandos}
\begin{mdframed}[backgroundcolor=gray!10]
\begin{lstlisting}[style=Raw]
java -jar simplesrh.jar
   --json funcionarios.json --listar
   --json funcionarios.json calcular-bonus
   --json funcionarios.json calcular-bonus --nome Maria
   --json funcionarios.json calcular-reajuste --percentual 10
   --json funcionarios.json calcular-reajuste --nome Junior --percentual 10
   --json funcionarios.json reajustar-salarios --cargo analista --percentual 10
   --json funcionarios.json --adicionar Bruno,Desenvolvedor,1000
   --json funcionarios.json --remover Bruno
\end{lstlisting}
\end{mdframed}

\subsubsection{Exemplo de uma lista de Funcionários}
\begin{mdframed}[backgroundcolor=gray!10]
\begin{lstlisting}[style=Raw]
[
    {
        "nome" : "Junior",
        "cargo" : "Desenvolvedor",
        "salario" : 1000.0,
        "dataAdmissao" : [ 2024, 10, 13 ],
        "pontosDesempenho" : 0
    },
    {
        "nome" : "Ana",
        "cargo" : "Analista",
        "salario" : 1320.0,
        "dataAdmissao" : [ 2023, 6, 1 ],
        "pontosDesempenho" : 80
    },
    ...
]
\end{lstlisting}
\end{mdframed}

\subsection{Estrutura diretórios SimplesRH}

\begin{mdframed}[backgroundcolor=gray!10]
\begin{lstlisting}[style=Raw]
    simplesrh
    |-- src/main/java/br/ufrn
    |   |-- simplesrh
    |   |   |-- command
    |   |   |   |-- CalcularBonusCommand.java
    |   |   |   |-- CalcularReajusteCommand.java
    |   |   |   |-- ReajustarCommand.java
    |   |   |-- dao
    |   |   |   |-- FuncionarioDAOImpl.java
    |   |   |   |-- FuncionarioDAO.java
    |   |   |-- model
    |   |   |   |-- Funcionario.java
    |   |   |-- service
    |   |   |   |-- FuncionarioService.java  // classe escolhida para os testes
    |   |   |   |-- GestaoService.java       // classe escolhida para os testes
    |   |   |-- SimplesRH.java
\end{lstlisting}
\end{mdframed}

As classes \colorbox{lightgray}{\texttt{\fontsize{12pt}{12pt}\selectfont FuncionarioService.java}} e \colorbox{lightgray}{\texttt{\fontsize{12pt}{12pt}\selectfont GestaoSerivice.java}} foram escolhidas, pois ambas contém as regras de negócio. Como adicionar um funcionário, caso o cargo seja um dos pré-definidos; calcular bônus de acordo com uma pontuação, etc.

\subsection{Injeção do bug}

\definecolor{lightlightgray}{rgb}{1, 1, 1}

Um bug de limite foi injetado na classe \colorbox{lightgray}{\texttt{\fontsize{12pt}{12pt}\selectfont GestaoSerivice.java}}

\begin{mdframed}[backgroundcolor=gray!10]
\begin{lstlisting}[language=Java]
public class GestaoService {

    private FuncionarioDAO funcionarioDAO;

    public GestaoService(FuncionarioDAO funcionarioDAO) {
        this.funcionarioDAO = funcionarioDAO;
    }

    public double calcularBonusPorDesempenho(Funcionario funcionario) {
        double salario = funcionario.getSalario();
        int pontosDesempenho = funcionario.getPontosDesempenho();
        
        double bonusPercentual = 0;
        if (pontosDesempenho >= 0 && pontosDesempenho <= 49) {
            bonusPercentual = 0.05;
        } else if (pontosDesempenho >= 50 && pontosDesempenho <= 69) {
            bonusPercentual = 0.10;
        } else if (pontosDesempenho >= 70 && pontosDesempenho <= 89) {
            bonusPercentual = 0.15;
                           // BUG INJETADO: pontosDesempenho == 100 não é avaliado
        } else if (pontosDesempenho >= 90 && pontosDesempenho < 100) {
            bonusPercentual = 0.20;
        } else {
            System.out.println("Pontuação de desempenho inválida. A pontuação deve ser entre 0 e 100.");
        }

        return salario * bonusPercentual;
    }

    public double calcularReajusteSalarial(Funcionario funcio, double percentual) {
        ...
    }

    public void reajustarSalarioPorCargo(String cargo, double percentualReajuste) {
        ...
    }
}

\end{lstlisting}
\end{mdframed}

\section{Testes Gerados}

\subsection{Randoop}

\subsection{Evosuite}

\end{document}

